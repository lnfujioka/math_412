\documentclass[12pt]{article}

%\usepackage[bbgreekl]{mathbbol}
\usepackage{amsmath}
\usepackage{amsthm}
\usepackage{amssymb}
\usepackage{amscd}
\usepackage{setspace}
\usepackage{mathrsfs}
\usepackage{color}
\usepackage{multirow}
\usepackage{stmaryrd}	%double brackets (for power series)
\usepackage{mathdots}	%for diagonal dots going up
\usepackage[unicode,urlcolor=blue,colorlinks=true]{hyperref}	%for hyperlinks
\usepackage{permute}
\renewcommand*\pmtseparator{\;}
\usepackage[margin=1in]{geometry}
\usepackage{tikz}
\usepackage{graphicx}
\usepackage{ulem}

%\newtheorem{theorem}{Theorem}[section]
%\newtheorem{proposition}[theorem]{Proposition}
%\newtheorem{lemma}[theorem]{Lemma}
%\newtheorem{corollary}[theorem]{Corollary}
%\newtheorem{claim}[theorem]{Claim}
%\newtheorem{question}[theorem]{Question}
%\newtheorem{conjecture}[theorem]{Conjecture}

\theoremstyle{definition}
\newtheorem*{definition}{Definition}
\newtheorem*{remark}{Remark}
%\newtheorem{example}[theorem]{Example}
%\newtheorem{assumption}[theorem]{Assumption}
%\newtheorem{recap}[theorem]{Recap}


\renewcommand{\phi}{\varphi}
\newcommand{\vect}[1]{\begin{pmatrix}#1\end{pmatrix}}
\newcommand{\mf}[1]{\mathfrak{#1}}
\newcommand{\wt}[1]{\widetilde{#1}}
\newcommand{\ol}[1]{\overline{#1}}
\newcommand{\mc}[1]{\mathcal{#1}}
\newcommand{\wh}[1]{\widehat{#1}}
\renewcommand{\emph}[1]{\textit{#1}}
\newcommand{\funcdef}[5]{\begin{array}{rcl} #1 : #2 & \rightarrow & #3 \\ #4 & \mapsto & #5 \end{array} }
\newcommand{\func}[4]{\begin{array}{rcl} #1 & \rightarrow & #2 \\ #3 & \mapsto & #4 \end{array} }
\newcommand{\QQ}{\mathbf{Q}}
\newcommand{\ZZ}{\mathbf{Z}}
\newcommand{\RR}{\mathbf{R}}
\newcommand{\CC}{\mathbf{C}}
\newcommand{\FF}{\mathbf{F}}
\newcommand{\GG}{\mathbf{G}}
\newcommand{\congmap}{\tilde{\rightarrow}}
\renewcommand{\mod}[1]{\text{ }(\mathrm{mod}\text{ }#1)}

\DeclareMathOperator{\Hom}{Hom}
\DeclareMathOperator{\Aut}{Aut}
\DeclareMathOperator{\id}{id}
\DeclareMathOperator{\lcm}{lcm}
\DeclareMathOperator{\Isom}{Isom}
\DeclareMathOperator{\sgn}{sgn}
\DeclareMathOperator{\ord}{ord}
\DeclareMathOperator{\Func}{Func}
\DeclareMathOperator{\SL}{SL}
\DeclareMathOperator{\GL}{GL}

\newcommand{\ds}{\displaystyle}

\onehalfspacing


%\hypersetup{urlcolor=blue}		%set color of links to blue

%\renewcommand{\phi}{\varphi}
\renewcommand{\labelenumi}{(\arabic{enumi})}
\renewcommand{\labelenumiii}{(\roman{enumiii})}

\newcommand{\headdd}[2]{\textbf{Assignment #1 -- #2 -- Math 412}}
\newcommand{\headddf}[3]{\textbf{Assignment #1 -- All #2 parts -- Math 412}\\ \mbox{}\\ \textbf{Due in class: Thursday, #3}}

\begin{document}

\begin{center}
%\headdd{1}{Part 1}
\headddf{1}{2}{Sept.\ 5, 2019}
\end{center}

\underline{Textbook exercises:}%\footnote{From Hungerford's \textit{Abstract algebra, An introduction, Third edition}}

\textbf{Section 1.1:} 6 (WI), 8 (WI)

\textbf{Section 1.2:} 6, 8, 11, 16, 30
\begin{section}{1.1.6}
	Since the quotient of $q$ divided by $c$ is $k$, we can express this in the form $q = ck + \alpha, 0 \leq \alpha < c.$  Notice now that $a = bq + r$ can be rewritten as $a = b(ck + \alpha) + r = bck + b\alpha + r = (bc)k + (b\alpha + r).$  Now if $b\alpha +r < bc$, we have the desired form and it is clear to see that $bc$ divides $a$ with quotient $k$.  If $b\alpha + r \geq bc$, since $b\alpha + r \in \mathbb{Z}$, we can rewrite $b\alpha + r = \beta bc + s$, with $0 \leq s < bc$. Then we have that $a = (bc)k + (b\alpha +r) = (bc)(k+\beta) + s$.  Again we find that $bc$ divides $a$ with quotient $k$.
\end{section}
\begin{section}{1.1.8}
	Suppose $\alpha \in \mathbb{Z}$.  We have that $\alpha$ must be either odd or even.  By using the Division Algorithm, we know that for $\alpha \in \mathbb{Z}$, we should be able to express any $\alpha$ in the form $\alpha = 2k$ or $\alpha = 2k + 1$.\\
	Suppose $\alpha$ is odd.  Then $\alpha = 2k + 1, k \in \mathbb{Z}$.  Moreover, $2\alpha = 4k + 2$, with $2\alpha$ even.  It follows that $2\alpha +1 = 4k + 3$ is odd.\\
	Suppose $\alpha$  is even.  Then $\alpha = 2k, k \in \mathbb{Z}$.  Moreover, $2\alpha = 4k$, and we have that $2\alpha + 1 = 4k + 1.$\\
	Again by using the Division Algorithm, we let $\alpha$ range over all $\mathbb{Z}$.  We can construct any odd integer and find that $2\alpha +1 = 4k + 1$ or $2 \alpha + 1 = 4k +3$, as required.
\end{section}
\begin{section}{1.2.6}
	If $a \mid b$, and $c \mid d$, there exist $k, j \in \mathbb{Z}$ such that $ka = b$ and $jc = d$.  By multiplying these two equations together we get that $bd = kajc = (ac)kj$.  $kj \in \mathbb{Z}$, so we can also say that $ac \mid bd$.
\end{section} 
\begin{section}{1.2.8}
	Let us name the common divisors of $n$ and $n + 1$ as $\alpha$.  We can then say that there exist $j, k \in \mathbb{Z}$ such that $\alpha k = n, \alpha j = n + 1$.  By subtracting, we get that $\alpha j - \alpha k = \alpha(j - k) = 1$  The two factors $(j - k)$ and $ \alpha$ must be integers, so the only ways to obtain $1$ are if $\alpha = j-k = \pm 1$.  It follows that $(n, n+1) = 1$.
\end{section}
\begin{section}{1.2.11}
	I will apply a similar approach to the above problem, taking $\alpha$ to be an arbitrary common divisor.
	\begin{subsection}{a}
		By assuming a common divisor, we have $j, k \in \mathbb{Z}$ such that $n = j \alpha, n + 2 = k \alpha$.  Then we have that $\alpha (k - j) = 2$.  Since these factors are integers and we only want to worry about the greatest divisor, let us ignore negative factors.  In this case we have that $\alpha (k-j) = 2 * 1$ or $ \alpha (k - j) = 1 *2.$  It follows that $(n, n+2) = 1$ or $(n, n +2) = 2$.
	\end{subsection}
\begin{subsection}{b}
	Using a similar construction to part a, we get that $6 = \alpha (k - j).$  The possible ways to make $6$ from positive integers are $1*6, 2*3, 3*2, 6*1$.  It follows that $(n, n+6) \in \{1, 2, 3, 6\}$.
\end{subsection}
\end{section}
\begin{section}{1.2.16 ???}
	If $(a,b) = d$, then $d$ divides both $a$ and $b$.  We can take $r, s \in \mathbb{Z}$ such that $a = ds, b=dr.$  Moreover, $a/d = s, b/d =r$.  By Theorem 1.2 we can say that $d = au + bv$ and $1 = \frac{a}{d}u + \frac{b}{d}v = su + rv$.
\end{section}
\begin{section}{1.2.30}
	The first step would be to prove that the set $S = \{a_1u_1 + ... + a_nu_n \vert u_i \in \mathbb{Z}\}$ is nonempty.  We take the sum $a_1^2 + ... + a_n^2 \geq 0$ since not all $a_i$ can be $0$ to prove that $S$ has an element.  It follows that we may use the Well-Ordering Axiom to get the smallest positive element of $S$, call it $t = a_1u_1 + ... + a_nu_n, a_i \in \mathbb{Z}$.  \\
	We may assume $a_i$ to be any of the integers contained in the sum.  By the Division Algorithm, we have that there exist $q_i, r_i \in \mathbb{Z}$ such that $a_i = tq_i + r_i , 0 \leq r_i < t$.  It follows that $r_i = a_i - tq_i = a_i - (a_1u_1 + ... + a_nu_n)q_i = a_i(1 - q_iu_i) + a_1(-u_iq_i) + ... + a_{i-1}(-u_{i-1}q_i) + a_{i + 1}(-u_{i+1}q_i) + ...  + a_n(-u_nq_i)$.  This is a linear combination of $a_i$'s, and by using the assumption that $t$ is the smallest positive element we can conclude that $r = 0$.  It follows that $t \vert a_i$.\\
	Let $c$ be another common divisor of $a_i$'s.  Then for some $k_i \in \mathbb{Z}$, we have $a_i = ck_i$.  It follows that $t = a_1u_1 + ... + a_nu_n = (ck_1)u_1 + ... + (ck_n)u_n = c(k_1u_1 + ... + k_nu_n)$.  So $c \vert t$, and $c \leq \|t\|$ but t is positive so $c \leq t$.\\
	Thus $t$ satisfies the conditions for being the gcd $d$.
\end{section}

\vspace{12pt}
\underline{Other exercises:}
\vspace{-4pt}
\begin{enumerate}
	\item Sums of three squares. In class, we showed that if $n$ has remainder $3$ when divided by $4$, then $n$ cannot be written as a sum of two squares. This question gives a similar result for sums of three squares.
	\begin{enumerate}
		\item Show that when divided by $8$, the square of any integer has remainder either $0,1$, or $4$.\\\\
		%%
			We can represent any integer, $n$, by the form $n = 8k + r$, with $k \in \mathbb{Z}, r \in \{0,1,2,3,4,5,6,7\}.$  Now I can compute.
		\begin{enumerate}
			\item $(8k + 0)^2 = 64k^2 = 8(8k^2) + 0$.
			\item $(8k + 1)^2 = 64k^2 + 16k + 1 = 8(8k^2 + 2k) + 1$.
			\item $(8k + 2)^2 = 64k^2 + 32k + 4 = 8(8k^2 + 4k) + 4$.
			\item $(8k + 3)^2 = 64k^2 + 48k + 9 = 8(8k^2 + 6k + 1) + 1$.
			\item $(8k + 4)^2 = 64k^2 + 64k + 16 = 8(8k^2 + 8k + 2) + 0$.
			\item $(8k + 5)^2 = 64k^2 + 80k + 25 = 8(8k^2 + 10k + 3) + 1$.
			\item $(8k + 6)^2 = 64k^2 + 96k + 36 = 8(8k^2 + 12k + 4) + 4$.
			\item $(8k + 7)^2 = 64k^2 + 112k + 49 = 8(8k^2 + 14k + 6) + 1$.
			
		\end{enumerate} 
		In all cases we have something of the form $8k + r$, where $r \in \{0,1,4\}$.\\
		%%
		\item Conclude that if $n$ has remainder $7$ when divided by $8$, then $n$ cannot be written as a sum of three squares.\\\\
		%%
		In class we used the fact that the sum of remainders (modulo $8$) is the remainder of the sum.  The above argument also proved that any number squared then divided by 8 must have a remainder of $0,1,4$.
		It is impossible to create a sum of $7$ from three numbers of the set $\{0,1,4\}$.  Thus, it is impossible for any number that is the sum of three squares to have a remainder of $7$ when divided by $8$.\\
		%%
		\item Give an example of a number of the form $4k+3$, $k\in\ZZ$, that \textit{is} a sum of three squares.\\\\
		%%
		A very simple example is
		\begin{equation}
		1^2 + 1^2 + 1^2 = 1 + 1 + 1 = 3 = 4*0 + 3.
		\end{equation}\\
		%%
	\end{enumerate}
	
	\item (WI) Let $a$ and $b$ be positive integers, and let $g=(a,b)$. Suppose $u_0,v_0\in\ZZ$ are such that $g=au_0+bv_0$.
	\begin{enumerate}
		\item Let $k\in\ZZ$ and let
		\[
			u=u_0+\frac{bk}{g}\quad\text{and}\quad v=v_0-\frac{ak}{g}.
		\]
		Show that $au+bv=g$.\\\\
		From the assumptions we get that $u_0 = u - \frac{bk}{g}, v_0 = v + \frac{ak}{g}$.  So $au_0 = au - \frac{bka}{g}, bv_0 = bv + \frac{bak}{g}$, and by adding these two equations we can see that  $g = au + bv$ as required. \\
		\item Conversely, show that if $g=au+bv$ with $u,v\in\ZZ$, then there is an integer $k$ such that
		\[
			u=u_0+\frac{bk}{g}\quad\text{and}\quad v=v_0-\frac{ak}{g}.
		\]
	\end{enumerate}
\end{enumerate}
\end{document}
\documentclass[12pt]{article}

%\usepackage[bbgreekl]{mathbbol}
\usepackage{amsmath}
\usepackage{amsthm}
\usepackage{amssymb}
\usepackage{amscd}
\usepackage{setspace}
\usepackage{mathrsfs}
\usepackage{color}
\usepackage{multirow}
\usepackage{stmaryrd}	%double brackets (for power series)
\usepackage{mathdots}	%for diagonal dots going up
\usepackage[unicode,urlcolor=blue,colorlinks=true]{hyperref}	%for hyperlinks
\usepackage{permute}
\renewcommand*\pmtseparator{\;}
\usepackage[margin=1in]{geometry}
\usepackage{tikz}
\usepackage{graphicx}
\usepackage{ulem}

%\newtheorem{theorem}{Theorem}[section]
%\newtheorem{proposition}[theorem]{Proposition}
%\newtheorem{lemma}[theorem]{Lemma}
%\newtheorem{corollary}[theorem]{Corollary}
%\newtheorem{claim}[theorem]{Claim}
%\newtheorem{question}[theorem]{Question}
%\newtheorem{conjecture}[theorem]{Conjecture}

\theoremstyle{definition}
\newtheorem*{definition}{Definition}
\newtheorem*{remark}{Remark}
%\newtheorem{example}[theorem]{Example}
%\newtheorem{assumption}[theorem]{Assumption}
%\newtheorem{recap}[theorem]{Recap}


\renewcommand{\phi}{\varphi}
\newcommand{\vect}[1]{\begin{pmatrix}#1\end{pmatrix}}
\newcommand{\mf}[1]{\mathfrak{#1}}
\newcommand{\wt}[1]{\widetilde{#1}}
\newcommand{\ol}[1]{\overline{#1}}
\newcommand{\mc}[1]{\mathcal{#1}}
\newcommand{\wh}[1]{\widehat{#1}}
\renewcommand{\emph}[1]{\textit{#1}}
\newcommand{\funcdef}[5]{\begin{array}{rcl} #1 : #2 & \rightarrow & #3 \\ #4 & \mapsto & #5 \end{array} }
\newcommand{\func}[4]{\begin{array}{rcl} #1 & \rightarrow & #2 \\ #3 & \mapsto & #4 \end{array} }
\newcommand{\QQ}{\mathbf{Q}}
\newcommand{\ZZ}{\mathbf{Z}}
\newcommand{\RR}{\mathbf{R}}
\newcommand{\CC}{\mathbf{C}}
\newcommand{\FF}{\mathbf{F}}
\newcommand{\GG}{\mathbf{G}}
\newcommand{\congmap}{\tilde{\rightarrow}}
\renewcommand{\mod}[1]{\text{ }(\mathrm{mod}\text{ }#1)}

\DeclareMathOperator{\Hom}{Hom}
\DeclareMathOperator{\Aut}{Aut}
\DeclareMathOperator{\id}{id}
\DeclareMathOperator{\lcm}{lcm}
\DeclareMathOperator{\Isom}{Isom}
\DeclareMathOperator{\sgn}{sgn}
\DeclareMathOperator{\ord}{ord}
\DeclareMathOperator{\Func}{Func}
\DeclareMathOperator{\SL}{SL}
\DeclareMathOperator{\GL}{GL}

\newcommand{\ds}{\displaystyle}

\onehalfspacing


%\hypersetup{urlcolor=blue}		%set color of links to blue

%\renewcommand{\phi}{\varphi}
\renewcommand{\labelenumi}{(\arabic{enumi})}
\renewcommand{\labelenumiii}{(\roman{enumiii})}

\newcommand{\headdd}[2]{\textbf{Assignment #1 -- #2 -- Math 412}}
\newcommand{\headddf}[3]{\textbf{Assignment #1 -- All #2 parts -- Math 412}\\ \mbox{}\\ \textbf{Due in class: Thursday, #3}}

\begin{document}

\begin{center}
%\headdd{2}{Part 1}
\headddf{2}{2}{Sept.\ 12, 2019}
\end{center}

\underline{Textbook exercises:}\footnote{From Hungerford's \textit{Abstract algebra, An introduction, Third edition}}

\textbf{Section 1.3:} 10 (WI), 17

\textbf{Section 2.1:} 6, 4

\textbf{Section 2.2:} 14 (14(c) will be WI)														

\begin{section}{1.3.10}
	First let us suppose that $p$ is prime.  Moreover we can express any element $a \in \mathbb{Z}$ as $a = p_1...p_n$, where $p_i$ are primes.  Now we have the prime factorization of $a$.  $p$ must either be in this prime factorization or not.
	\begin{itemize}
		\item Suppose there exists some $p_k$ such that $p = p_k$.  We have that $a = p_k(p_1...p_{k-1}p_{k+1}...p_n)$.  Here we can see that $p \mid a$.
		\item Suppose that there is no $p_i$ such that $p = p_i$.  Since we have assumed that $p$ is prime we know that if $p$ itself is not in the factorization of $a$, they will not share any common factors.  It follows that $(a,p) = 1$.  
	\end{itemize}

	\noindent Now let us suppose that we have $p$ is not prime.  Then we can express $p$ as a product of primes.  $p = p_1...p_n$.  We can show that given $p$, we can construct $a$ such that $(a,p) > 1$ and $p \nmid a$.  Say $a = p_1...p_{n-1}$ so $(a, p) = a$.  Then we have that $p > a$ so $p \nmid a$.
\end{section}
\begin{section}{1.3.17 do later}
	If $(a,b) = p$ we can say that $(a^2, b^2) = p^2$.  We can factorize $a,b$ and order the primes, $q_i$ in increasing order, remembering that $p$ is a factor of both.  $a = q_11...q_{k}pq_{k + 1}...q_n$, $b = q_1$
\end{section}
\begin{section}{2.1.6}
	Suppose that $a \equiv b \mod{n}$ and $k \mid n$.  We can say that there exists $\alpha$ such that $n = k\alpha$.  Also by definition of mod, $n \mid a-b$, so there exists $\beta$ such that $a-b = \beta n$.  We have that $a - b = \beta n = \beta k \alpha = (\alpha\beta)k$.  So $k \mid a-b$ and it follows that $a \equiv b \mod{k}$.
\end{section}


\vspace{12pt}
\underline{Other exercises:}
\vspace{-4pt}
\begin{enumerate}
	\item (WI) Prime factorization, gcds, and lcms. Let $a$ and $b$ be non-zero integers and let $p_1,\dots,p_k>0$ be the positive primes that divide $a$ or $b$ (or both!). Write
	\[
		a=up_1^{e_1}p_2^{e_2}\cdots p_k^{e_k}\quad\text{and}\quad b=vp_1^{f_1}p_2^{f_2}\cdots p_k^{f_k}
	\]
	with $u,v\in\{\pm1\}$ and $e_i,f_i\geq0$ (recall from class that $u,v$, and the $e_i$ and $f_i$ are unique).
	\begin{enumerate}
		\item Show that $a\mid b$ if and only if $e_i\leq f_i$ for all $i=1,2,\dots,k$.
		\item Show that
		\[
			\gcd(a,b)=p_1^{g_1}p_2^{g_2}\cdots p_k^{g_k}
		\]
		where $g_i=\min(e_i,f_i)$.
		\item The \textit{least common multiple} of $a$ and $b$ is the smallest positive integer that is divisible by both $a$ and $b$. It is denoted $\lcm(a,b)$ or simply $[a,b]$ (like the $\gcd$ is sometimes denoted simply $(a,b)$). Show that
		\[
			\lcm(a,b)=p_1^{h_1}p_2^{h_2}\cdots p_k^{h_k}
		\]
		where $h_i=\max(e_i,f_i)$.
		
		\item Show that $ab=(a,b)[a,b]$.
	\end{enumerate}
	
	\item Suppose $p$ is a prime number and $a\in\ZZ$.
	\begin{enumerate}
		\item Let $n\in\ZZ_{\geq0}$. Show that if $p\mid a^n$, then $p^n\mid a^n$.
		\item Let $e\in\ZZ_{\geq0}$. We say that $p^e$ \textit{exactly divides} $a$ (written $p^e\mid\mid a$) if $p^e\mid a$ and $p^{e+1}\nmid a$. Let $b\in\ZZ$. Let $b\in\ZZ$ and suppose $p^e\mid\mid a$ and $p^f\mid\mid b$. Show that $p^{e+f}\mid\mid ab$.
		\item Suppose $p^e\mid\mid a$ and $p^f\mid\mid b$. Show $p^{\min(e,f)}\mid b$. Show by example that it can happen that $p^{\min(e,f)}$ does not exactly divide $a+b$.
	\end{enumerate}
	
	\item Let $n\in\ZZ_{\geq1}$. Show that for $a\in\ZZ$, its congruence class modulo $n$ is given by
	\[
		[a]=\{a+nk:k\in\ZZ\}.
	\]
	
	\item Write out addition and multiplication tables for $\ZZ/4\ZZ$. (The tables for $\ZZ/5\ZZ$ and $\ZZ/6\ZZ$ are in example 2 of \S2.2 of the textbook.)
\end{enumerate}
\end{document}